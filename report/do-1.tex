\documentclass[11pt,a4paper,english]{article}
\usepackage[utf8]{inputenc}
\usepackage{babel}
\usepackage[T1]{fontenc}
\usepackage{lmodern}

\usepackage{amsmath}
\usepackage{amssymb}
\usepackage{amsthm}
\usepackage{bm}

\begin{document}
\title{Discrete Optimization Assignment 1}
\author{Names}
\maketitle
\tableofcontents
\clearpage

\section{Theoretical part - formulation and lower bounds}
Given a complete directed graph $G = (V, E, c)$ where $V$ is the set of vertices with
$|V| = n$, $E$ is the set of edges, and $c_{ij}$ is the cost
of the edge $(i,j) \in E$, then the asymmetric TSP problem is defined as finding a
 Hamiltonian tour of $G$ which has minimum total cost.
 \subsection{}
 Equation (\ref{eq:tsp:subtour}) defines the  ``subtour''  formulation
  for the asymmetric TSP problem  an Integer
  Linear Program.
 \begin{equation}
   \label{eq:tsp:subtour}
   \begin{aligned}
     & \underset{x_{ij}}{\text{minimize}}
     & & \sum_{i,j \in V} x_{ij}c_{ij} \\
     & \text{subject to}
     & & \sum_{j \in V, i \neq j} x_{ij} = 1 \quad i \in V \\
     & & & \sum_{i \in V, i \neq j} x_{ij} = 1 \quad j \in V \\
     & & & \sum_{i,j \in S} x_{ij} \leq |S| - 1 \quad S \subset V, 2 \leq |S| \leq n-2 \\
     & & & x_{ij} \in \{0,1\}
   \end{aligned}
 \end{equation}
 We wish to show that this formulation is correct, that is, that a TSP solution is feasible if and only if it satisfies the ILP constraints.
 \begin{proof}
   We will prove this by construction in two cases; first
 we will show that given a solution, $S_{TSP}$, we can construct an equivalent
 feasible solution $S_{ILP}$ to the ILP, and subsequently the other way around.
 
 \paragraph{Case 1 ($S_{TSP} \to S_{ILP}$)}
 A feasible solution the the TSP problem consists of a Hamiltonian tour, which is a permutation of the vertices $V$ or, in other words, a cyclic path in $G$ which visits
  every vertex exactly once. We denote this permutation as
  $$S_{TSP} = (p_1, p_2, \dots, p_n )$$
  % use parentheses (ordered set) instead of brackets (set) to indicate that the order matters
  where each $p_i$ bijectively maps to a vertex in $V$. The permutation defines the path
  $$p^* = \left\{ (p_1, p_2), (p_2, p_3),\ldots,(p_{n-1}, p_{n}), (p_n, p_1) \right\}$$
  where $p_i \neq p_j$ for $i\neq j$ follows directly from $S_{TSP}$
   being a Hamiltonian tour. This solution has the total cost
  $$C_{{TSP}} = c_{p_1p_2} + \dots + c_{p_{n-1}p_{n}} + c_{p_{n}p_1}\mathnormal{.}$$
  We then construct a solution, $S_{ILP}$, to the ILP as follows,
  $$x_{ij} =
  \begin{cases}
    1 & (i,j) \in p^* \\
    0 & \text{otherwise.}
  \end{cases}
  $$
  It follows directly from this assignment that the ILP
  solution has the same cost as the TSP solution,
$$C_{ILP} = \sum_{i,j \in V} x_{ij}c_{ij} = \sum_{(i,j) \in p^*} 1c_{ij} + \sum_{(i,j) \not\in p^*} 0c_{ij} =  \sum_{(i,j) \in p^*} c_{ij} = C_{{TSP}}\mathnormal{.}$$
It is then left to show that all the constraints of (\ref{eq:tsp:subtour}) are
satisfied. Since $S_{TSP}$ is a permutation of $V$ we can equivalently state
the first constraint as
$$\sum_{j \in V, i \neq j} x_{p_ip_j} = 1 \quad i \in V$$
and note that for any $i \in V$
$$\sum_{j \in V, i \neq j} x_{p_ip_j} = \underbrace{\ldots + x_{p_i p_{i-1}}}_{=0} +
\underbrace{x_{p_i p_{i+1}}}_{=1} + \underbrace{x_{p_i p_{i+2}} + \ldots}_{=0} = 1\mathnormal{.}$$
This follows directly from $p^*$ and our assignment of $x_{ij}$ (where we technically should have $i+1 \mod n$). Similarly, for any $j \in V$, we have
$$\sum_{i \in V, i \neq j} x_{p_ip_j} = \underbrace{\ldots + x_{p_{j-2} p_j}}_{=0} +
\underbrace{x_{p_{j-1} p_j}}_{=1} + \underbrace{x_{p_{j+1} p_j} + \ldots}_{=0} = 1\mathnormal{.}$$
% could be i and j instead of p_i and p_j, but this makes it clearer, maybe. I don't think it would be an equivalent formulation to write \sum_{i,j \in S} x_{p_ip_j} <= ... since there then wouldn't be any direct relationship between i and p_i, and j and p_j.
Hence the first two constraints are satisfied.

For the subtour constraint, we once again note that we can equivalently state
the constraint in terms of our permutation $S_{TSP}$, and we note that we
\textit{only} have $x_{p_i p_j} = 1$ if $j = i + 1 \mod n$.
Hence, given a subset $S \subset V$ with size $|S| < n$, we can have at most
$|S| - 1$ pairs $(p_i, p_j)$, which means we have
$$\sum_{p_i,p_j \in S} x_{p_i p_j} \leq |S| - 1 \quad S \subset V, 2 \leq |S| \leq n-2\mathnormal{.}$$
Finally, the $\{0,1\}$ constraint is fulfilled directly by the
assignment of $x_{ij}$.
 \paragraph{Case 2 ($S_{ILP} \to S_{TSP}$)}
 Given a solution, $S_{ILP}$, to (\ref{eq:tsp:subtour}) which consists of an assignment of $\bm X = x_{ij}$, we want to construct a TSP solution $S_{TSP}$. For it to be feasible, it must be a permutation of $V$. We construct it as follows:
 \begin{enumerate}
 \item Set $S_{TSP} = \emptyset$ and $i = 1$.
 \item Traverse the $i$th row of $\bm X$ from $1,\dots,n$ until a variable $x_{i,j} = 1$ is found.
 \item Set $S_{TSP} = S_{TSP} \cup \{i\}$ and $i = j$.
 \item If $j = 1$ return $S_{TSP}$; else go to step 2.
 \end{enumerate}
 The first constraint of (\ref{eq:tsp:subtour}) ensures that each row, $i$,
 of $\bm X$ will
 have exactly one $x_{i,j} = 1$ which implies that step 2 will always find a
 variable $x_{i,j} = 1$. Likewise the second constraint ensures that step 4 will
 go to each row at most once. Hence we have $p_i \neq p_j$  if $i \neq j$
 for all $p_i, p_j \in S_{TSP}$, i.e. all added vertices are unique.

 For $S_{TSP}$ to define a permutation of $V$, we must show that $|S_{TSP}| = n$.
We prove by contradiction that $|S_{TSP}| \geq n$. Let $S_{TSP}$ be constructed from a feasible ILP solution $S_{ILP}$ and assume $|S_{TSP}| = m, m < n$. By the construction procedure, $S_{TSP}$ must consist of vertices $s_1,s_2,\dots,s_m$ such that
 $$x_{s_1 s_2} = x_{s_2 s_3} = \ldots = x_{s_{m-1} s_m} =  x_{s_{m} s_1} =  1\mathnormal{.}$$
Then we have
 $$\sum_{s_i, s_j\in S_{TSP}} x_{s_i s_j} = x_{s_1 s_2} + x_{s_2 s_3} + \ldots + x_{s_{m-1} s_m} +  x_{s_{m} s_1} = |S_{TSP}|.$$
If $m \leq n-2$ then this violates the subtour constraint which is in contradiction
  with $S_{ILP}$ being a feasible solution. If $m = n -1$ we
 must have that there exists a single unvisited row, $k$, which only has
 $x_{k,\cdot} = 1$ on $x_{kk}$, which would violate the first constraint which is
 also a contradiction.
Therefore $m = |S_{TSP}| \geq n$.  As all vertices are unique, $|S_{TSP}| \leq n$, and we have $|S_{TSP}| = n$.

 Hence $S_{TSP}$ must be a permutation of $V$ defining a Hamiltonian tour. Since
 we selected $i,j \in S_{TSP}$ iff. $x_{ij} = 1$ we will also have that
  the path defined by $S_{TSP}$ will have the same cost as $S_{ILP}$.
 \end{proof}

\subsection{}
From Equation (1) in 1.1 we have the definition of the "sub-tour" formulation for the TSP problem an ILP. From the sub-tour constraint we have 
$$\sum_{i,j \in S} x_{p_i p_j} \leq |S| - 1 \quad S \subset V, 2 \leq |S| \leq n-2\mathnormal{.}$$
From this we can determine the number of constraints in terms of n.
The number of possible sets in V can be found to be $2^n$. 
The constraints consist of all possible sets in V devoid of the S = V, the empty set, all sets of size one and sets of size n-1. 

Because there are as many sets of size one and of size n-1 as there are vertices, it follows that these subsets that cannot be included in the constraints amount to 2n. 

As well the empty and full set are unique sets and together amount to 2. 
Thus the number of constraints in terms of n can be formulated to be $$2^n - 2(n+1)$$.

\subsection{}
Equation (2) represents the compact formulation.  
 \begin{equation}
   \label{eq:tsp:subtour}
   \begin{aligned}
     & \underset{x_{ij}}{\text{minimize}}
     & & \sum_{i,j \in V} x_{ij}c_{ij} \\
     & \text{subject to}
     & & \sum_{j \in V, i \neq j} x_{ij} = 1 \quad i \in V \\
     & & & \sum_{i \in V, i \neq j} x_{ij} = 1 \quad j \in V \\
     & & & t_{j} \geq t_{i} + 1 - n(1-x_{ij} )\\
     & & & i\in V, j\in V \backslash \{1\} \\
     & & & x_{ij} \in \{0,1\}\\
     & & & t_{i}\in\mathbb{R}_+
   \end{aligned}
 \end{equation}
 
 Let $t_{i}$ be integers or real numbers representing the order in which the vertices are visited. Thus if $t_{i} = 1$, vertex $i$ is the first vertex visited.
The constraints are of the form $$t_{j} \geq t_{i} + 1 - n(1-x_{ij} )$$ The compact formulation has fewer constraints than the constraints for Equation (1).
In terms of n there are n*(n-1) constraints.

The constraint states if there is an edge from vertex $i$ to $j$ such that $x_{ij} = 1$ the value of $t_{j}$ must be at least one greater than $t_{i}$. Thus the formulation has constraints consisting of $n$ possible values for $t_{i}$ and $(n-1)$ possible values for $t_{j}$ because the 1 vertex is not included in the $j$ set. For this reason the number of constraints in terms of n can be represented by $$n(n-1)$$

\subsection{}
Why subtour formulation might be preferred over compact formulation.
Because the added t variable make the linear relaxation much weaker than the subtour. Further explanation needed.

\subsection{}
We consider the subset of edges of $G$ consisting of a tree together with an additional edge incident to vertex 1 such that vertex 1 is in a cycle. For reference, we call this a 1-cycle. We wish to show that the minimum cost of the 1-cycle is a lower bound for for the minimum cost Hamiltonian tour.

\begin{proof}
Let $S$ be a minimum cost Hamiltonian tour and $T$ be a minimum cost 1-cycle of a graph $G$. We need to show that $c(S) \geq c(T)$.

Vertex 1 is connected to $S$ with two edges. If we remove one of these two edges, $e$, we have a tree. If we add the shortest edge $e'$ incident to vertex 1, we have a 1-cycle $T'$. Since $c(e) \geq c(e')$, then $c(S) \geq c(T') \geq c(T)$.
\end{proof}

Intuitively, the 1-cycle is similar to the 1-tree, but is a lower bound for the 1-tree, since any 1-tree is a 1-cycle (but not vice versa). As the 1-tree is also a lower bound for the TSP, it is thus a stronger lower bound than 1-cycle.
\clearpage

\section{Implementation part - branch-and-bound}

\subsection{}
Upper bound, heuristic.

\subsection{}
Implementation.

\subsection{}
Running time and number of nodes.

\end{document}