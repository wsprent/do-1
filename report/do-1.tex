\documentclass[11pt,a4paper,english]{article}
\usepackage[utf8]{inputenc}
\usepackage{babel}
\usepackage[T1]{fontenc}
\usepackage{lmodern}

\usepackage{amsmath}
\usepackage{amssymb}
\usepackage{amsthm}
\usepackage{bm}

\begin{document}
\title{Discrete Optimization Assignment 1}
\author{Names}
\maketitle
\tableofcontents
\clearpage

\section{Theoretical part - formulation and lower bounds}
Given a complete graph $G = (V, E, c)$ where $V$ is the set of vertices with
$|V| = n$, $E$ is the set of edges, and $c_{ij}$ is the cost
of the edge $(i,j) \in E$, then the TSP problem is defined as finding a
 Hamiltonian tour of $G$ which has minimum total cost.
 \subsection{}
 Equation (\ref{eq:tsp:subtour}) defines the  ``subtour''  formulation
  for the TSP problem  an Integer
  Linear Program.
 \begin{equation}
   \label{eq:tsp:subtour}
   \begin{aligned}
     & \underset{x_{ij}}{\text{minimize}}
     & & \sum_{i,j \in V} x_{ij}c_{ij} \\
     & \text{subject to}
     & & \sum_{j \in V, i \neq j} x_{ij} = 1 \quad i \in V \\
     & & & \sum_{i \in V, i \neq j} x_{ij} = 1 \quad j \in V \\
     & & & \sum_{i,j \in S} x_{ij} \leq |S| - 1 \quad S \subset V, 2 \leq |S| \leq n-2 \\
     & & & x_{ij} \in \{0,1\}
   \end{aligned}
 \end{equation}
 We wish to show that this formulation is correct. That is that any feasible
 solution to the TSP problem is a feasible solution to
 the ILP.
 \begin{proof}
   We will prove this by construction in two cases; first
 we will show that given a solution, $S_{TSP}$, we can construct an equivalent
 feasible solution $S_{ILP}$ to the ILP, and subsequently the other way around.
 
 \paragraph{Case 1 ($S_{TSP} \to S_{ILP}$)}
 A feasible solution the the TSP problem consists of a Hamiltonian tour, that
 is a permutation of the vertices $V$ or in other words a path in $G$ which visits
  every vertice exactly one. We then denote this path as,
  $$S_{TSP} = \{p_1, p_2, ..., p_n \}$$
 whichs defines the path
  $$p^* = \left\{ (p_1, p_2), (p_2, p_3),\ldots,(p_{n-1}, p_{n}), (p_n, p_1) \right\}$$
  where $p_i \neq p_j$ for $i\neq j$ follows directly from $S_{TSP}$
   being a Hamiltonian tour. This solution has the total cost
  $$C_{{TSP}} = c_{p_1p_2} + ... + c_{p_{n-1}p_{n}} + c_{p_{n}p_1}\mathnormal{.}$$
  We then construct a solution, $S_{ILP}$, to the ILP as follows,
  $$x_{ij} =
  \begin{cases}
    1 & (i,j) \in p^* \\
    0 & \text{otherwise.}
  \end{cases}
  $$
  It follows directly from this assignment that the ILP
  solution has the same cost as the TSP solution,
$$C_{ILP} = \sum_{i,j \in V} x_{ij}c_{ij} = \sum_{(i,j) \in p^*} 1c_{ij} + \sum_{(i,j) \not\in p^*} 0c_{ij} =  \sum_{(i,j) \in p^*} c_{ij} = C_{{TSP}}\mathnormal{.}$$
It is then left to show that all the constraints of (\ref{eq:tsp:subtour}) are
satisfied. Since $S_{TSP}$ is a permutation of $V$ we can equivalently state
the first constraint as
$$\sum_{j \in V, i \neq j} x_{p_ip_j} = 1 \quad i \in V$$
and note that
$$\sum_{j \in V, i \neq j} x_{p_ip_j} = \underbrace{\ldots + x_{p_i p_i}}_{=0} +
\underbrace{x_{p_i p_{i+1}}}_{=1} + \underbrace{x_{p_i p_{i+2}} + \ldots}_{=0} = 1\mathnormal{.}$$

follows directly from $p^*$ and our assignment of $x_{ij}$ (where we technically
 have $i+1 = (i \mod n) + 1$). Similarly, we have
$$\sum_{i \in V, i \neq j} x_{p_ip_j} = \underbrace{\ldots + x_{p_{j-1} p_j}}_{=1} +
\underbrace{x_{p_{j} p_j} + x_{p_{j+2} p_j} + \ldots}_{=0} = 1$$
Hence the first two constraints are satisfied.

For the subtour constraint, we once again note that we can equivalently state
the constraint in terms of our permutation $S_{TSP}$, and we note that we
\textit{only} have $x_{p_i p_j} = 1$ if either $j = i + 1$ or $i = n$ and $j = 1$.
Hence, given a subset $S \subset V$ with size $|S|$, we can have at most
$|S| - 1$ pairs $(i, i+1)$, which means we have
$$\sum_{i,j \in S} x_{p_i p_j} \leq |S| - 1 \quad S \subset V, 2 \leq |S| \leq n-2\mathnormal{.}$$
Finally, the $\{0,1\}$ constraint is fulfilled directly by the
assignment of $x_{ij}$.
 \paragraph{Case 2 ($S_{ILP} \to S_{TSP}$)}
 Given a solution, $S_{ILP}$, to (\ref{eq:tsp:subtour}) which consists of an assignment of $\bm X = x_{ij}$, we can construct a hamiltonian tour as follows:
 \begin{enumerate}
 \item Set $S_{TSP} = \emptyset$ and $i = 1$.
 \item Traverse the $i$th row of $\bm X$ from $1,...,n$ until a variable $x_{i,j} = 1$ is found.
 \item Set $S_{TSP} = S_{TSP} \cup \{i\}$ and $i = j$.
 \item If $j = 1$ return $S_{TSP}$; else go to step 2.
 \end{enumerate}
 The first constraint of (\ref{eq:tsp:subtour}) ensures that each row, $i$,
 of $\bm X$ will
 have exactly one $x_{i,j} = 1$ which implies that step 2 will always find a
 variable $x_{i,j} = 1$. Likewise the second constraint ensures that step 4 will
 go to each row at most once. Hence we have $p_i \neq p_j$  if $i \neq j$
 for all $p_i, p_j \in S_{TSP}$.

 To show that $S_{TSP}$ defines a permutation of $V$, we must then show that
 $|S_{TSP}| = |V|$. The only way we can have $|S_{TSP}| < |V|$ is if there
 exists a sequence $s_1, s_2, \ldots, s_m$ where $m < n$ such that
 $$x_{s_1 s_2} = x_{s_2 s_3} = \ldots = x_{s_{m-1} s_m} =  x_{s_{m} s_1} =  1\mathnormal{.}$$
 Let $S = \{s_1, s_2, \ldots, s_m \}$ then we have
 $$\sum_{s_i, s_j\in S} x_{s_i s_j} + x_{s_1 s_2} + x_{s_2 s_3} + \ldots + x_{s_{m-1} s_m} +  x_{s_{m} s_1} = |S|$$
 if $m \leq n-2$ then this violates the subtour constraint which is in contradiction
  with $S_{ILP}$ being a feasible solution. If $m = n -1$ we
 must have that there exists a single unvisited row, $k$, which only has
 $x_{k,\cdot} = 1$ on $x_{kk}$, which would violate the first constraint which is
 also a contradiction.

 Hence $S_{TSP}$ must a permutation of $V$ defining a Hamiltonian tour. Since
 we selected $p_i \in S_{TSP}$ iff. $x_{p_{i-1} p_i} = 1$ we will also have that
  the path defined by $S_{TSP}$ will have the same cost as $S_{ILP}$.
 \end{proof}

\subsection{}
From Equation (1) in 1.1 we have the definition of the "sub-tour" formulation for the TSP problem an ILP. From the sub-tour constraint we have 
$$\sum_{i,j \in S} x_{p_i p_j} \leq |S| - 1 \quad S \subset V, 2 \leq |S| \leq n-2\mathnormal{.}$$
From this we can determine the number of constraints in terms of n.
The number of possible sets in V can be found to be $2^n$. 
The constraints consist of all possible sets in V devoid of the S = V, the empty set, all sets of size one and sets of size n-1. 

Because there are as many sets of size one and of size n-1 as there are vertices, it follows that these subsets that cannot be included in the constraints amount to 2n. 

As well the empty and full set are unique sets and together amount to 2. 
Thus the number of constraints in terms of n can be formulated to be $$2^n - 2(n+1)$$.

\subsection{}
Equation (2) represents the compact formulation.  
 \begin{equation}
   \label{eq:tsp:subtour}
   \begin{aligned}
     & \underset{x_{ij}}{\text{minimize}}
     & & \sum_{i,j \in V} x_{ij}c_{ij} \\
     & \text{subject to}
     & & \sum_{j \in V, i \neq j} x_{ij} = 1 \quad i \in V \\
     & & & \sum_{i \in V, i \neq j} x_{ij} = 1 \quad j \in V \\
     & & & t_{j} \geq t_{i} + 1 - n(1-x_{ij} )\\
     & & & i\in V, j\in V \backslash \{1\} \\
     & & & x_{ij} \in \{0,1\}\\
     & & & t_{i}\in\mathbb{R}_+
   \end{aligned}
 \end{equation}
 
 Let $t_{i}$ be integers or real numbers representing the order in which the vertices are visited. Thus if $t_{i} = 1$, vertex $i$ is the first vertex visited.
The constraints are of the form $$t_{j} \geq t_{i} + 1 - n(1-x_{ij} )$$ The compact formulation has fewer constraints than the constraints for Equation (1).
In terms of n there are n*(n-1) constraints.

The constraint states if there is an edge from vertex $i$ to $j$ such that $x_{ij} = 1$ the value of $t_{j}$ must be at least one greater than $t_{i}$. Thus the formulation has constraints consisting of $n$ possible values for $t_{i}$ and $(n-1)$ possible values for $t_{j}$ because the 1 vertex is not included in the $j$ set. For this reason the number of constraints in terms of n can be represented by $$n(n-1)$$

\subsection{}
Why subtour formulation might be preferred over compact formulation.
Because the added t variable make the linear relaxation much weaker than the subtour. Further explanation needed.

\subsection{}
Proof lower bound.

\clearpage

\section{Implementation part - branch-and-bound}

\subsection{}
Upper bound, heuristic.

\subsection{}
Implementation.

\subsection{}
Running time and number of nodes.

\end{document}